\documentclass[a4paper]{report}

\usepackage[utf8]{inputenc}
\usepackage[T1]{fontenc}
\usepackage[francais]{babel}
\usepackage[colorlinks=true, linkcolor=blue, citecolor=red]{hyperref}

\title{Les objets connectés}
\author{Bastien \bsc{Labouche}\\Julie \bsc{Lopez}}
\date{24 février 2018}

\begin{document}
	
	\maketitle
	
	\tableofcontents
	
	\newpage
	
	\section{Les objets connectés}
	\subsection{Objets connectés ? Kesako ?}
	Si l'on se base sur la \href{http://objets.insa-rennes.fr/objets-connectes/quest-ce-quun-objet-connecte/}{définition de l'Insa Rennes} :
	\begin{quotation}
		un objet connecté est une chose, fabriquée par l’homme, dont l’usage est d’établir une liaison afin de pouvoir faire passer des 					informations diverses et variées à un autre objet ou à toute autre chose connectée.
	\end{quotation}
	Il s'agit là d'une définition très basique mais assez claire de ce qu'est un objet connecté, même si il n'y a pas vraiment de définition
	"officielle" de ce qu'est un objet connecté. Nous retiendrons donc que ce sont des objets pouvant "communiquer", et ce grâce à différentes
	technologies, par exemple :
	
	\bigbreak
	
	\begin{itemize}
		\item La montre connectée communiquant avec le téléphone grâce au bluetooth.
		\item La caméra IP communiquant via un câble ethernet avec la box.
		\item Le drone piloté avec le téléphone via du wifi.
		\item Le thermostat compatible wifi.
		\item ...
	\end{itemize}

	\bigbreak
	
	Au fur et à mesure de l'avancée de la technologie, une sorte de réseau entre les objets s'est développé, on l'appel "l'internet des objets" 	mais vous le connaissez sûrement sous le nom de IoT (Internet of Things)
	
	\newpage
	
	\subsection{Internet Of Things}
	
	Comme dit plus haut, l'Internet of Things est un réseau reliant entre eux les objets connectés de par le monde, il s'agit de leur version
	d'internet. \href{https://fr.wikipedia.org/wiki/Internet_des_objets}{Selon Wikipedia}, il s'agit d'une extension d'internet qui serait
	considéré comme la troisième évolution de l'internet :
	
	\begin{quotation}
		L'Internet des objets, ou IdO (en anglais Internet of Things, ou IoT), est l'extension d'Internet à des choses et à des lieux du monde
		physique.

		Alors qu'Internet ne se prolonge habituellement pas au-delà du monde électronique, l'Internet des objets connectés représente les
		échanges d'informations et de données provenant de dispositifs du monde réel avec le réseau Internet.
	\end{quotation}
	
	\bigbreak
	
	
	
	
	\section{Failles de sécurité}
	\subsection{Problèmes de configuration}
	\subsection{Failles systèmes}
	
	\section{Pourquoi c'est dangereux}
	\subsection{BotNet}
	\subsection{Souriez, vous êtes filmés}
	
	\section{Conclusion}
\end{document}