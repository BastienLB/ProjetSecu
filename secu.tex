\documentclass[a4paper]{report}

\usepackage[utf8]{inputenc}
\usepackage[T1]{fontenc}
\usepackage[francais]{babel}
\usepackage[colorlinks=true, linkcolor=blue, citecolor=red]{hyperref}

\title{Les objets connectés}
\author{Bastien \bsc{Labouche}\\Julie \bsc{Lopez}}
\date{24 février 2018}

\begin{document}
	
	\maketitle
	
	\tableofcontents
	
	\newpage
	
	\section{Les objets connectés}
	\subsection{Objets connectés ? Kesako ?}
	Si l'on se base sur la \href{http://objets.insa-rennes.fr/objets-connectes/quest-ce-quun-objet-connecte/}{définition de l'Insa Rennes} :
	\smallbreak
	\begin{quotation}
		un objet connecté est une chose, fabriquée par l’homme, dont l’usage est d’établir une liaison afin de pouvoir faire passer des 					informations diverses et variées à un autre objet ou à toute autre chose connectée.
	\end{quotation}
	\medbreak
	Il s'agit là d'une définition très basique mais assez claire de ce qu'est un objet connecté, même si il n'y a pas vraiment de définition
	"officielle" de ce qu'est un objet connecté. Nous retiendrons donc que ce sont des objets pouvant "communiquer", et ce grâce à différentes
	technologies, par exemple :
	
	\bigbreak
	
	\begin{itemize}
		\item La montre connectée communiquant avec le téléphone grâce au bluetooth.
		\item La caméra IP communiquant via un câble ethernet avec la box.
		\item Le drone piloté avec le téléphone via du wifi.
		\item Le thermostat compatible wifi.
		\item ...
	\end{itemize}

	\bigbreak
	
	Au fur et à mesure de l'avancée de la technologie, une sorte de réseau entre les objets s'est développé, on l'appel "l'internet des objets" 	mais vous le connaissez sûrement sous le nom de IoT (Internet of Things)
	
	\newpage
	
	\subsection{Internet Of Things}
	
	Comme dit plus haut, l'Internet of Things est un réseau reliant entre eux les objets connectés de par le monde, il s'agit de leur version
	d'internet. \href{https://fr.wikipedia.org/wiki/Internet_des_objets}{Selon Wikipedia}, il s'agit d'une extension d'internet qui serait
	considéré comme la troisième évolution de l'internet :
	\smallbreak
	\begin{quotation}
		L'Internet des objets, ou IdO (en anglais Internet of Things, ou IoT), est l'extension d'Internet à des choses et à des lieux du monde
		physique.

		Alors qu'Internet ne se prolonge habituellement pas au-delà du monde électronique, l'Internet des objets connectés représente les
		échanges d'informations et de données provenant de dispositifs du monde réel avec le réseau Internet.
	\end{quotation}
	
	\bigbreak
	
	L'Union internationale des télécommunications nous donne la définition suivante, décrivant l'internet des objets comme une :
	\begin{quotation}
		infrastructure mondiale pour la société de l'information, qui permet de disposer de services évolués en interconnectant 
		des objets(physiques ou virtuels) grâce aux technologies de l'information et de la communication interopérables existantes 
		ou en évolution
	\end{quotation}
	
	\medbreak	
	
	\paragraph{Quelques chiffres}
	\smallbreak
	\begin{itemize}
		\item En 2016, 5.5 millions d'objets sont connectés chaque jour dans le monde
		\item Gartner inc. prévoit que 26 milliards d'objets seront installés d'ici 2020
		\item Un être humain serait en contact avec 1000 à 5000 objets au cours d'une journée normale
		\item Il ne faut que quelques minutes pour qu'un objet connecté vulnérable se fasse hacker après sa mise en ligne
	\end{itemize}
	
	\newpage	
	
	\section{Failles de sécurité}
	\subsection{Problèmes de configuration}
	Comme vu plus haut dans la partie sur l'IoT des millions (et bientôt des milliards) d'objets sont interconnectés entre eux. Même si
	de nos jours la sécurité des données commence à entrer dans les mœurs, 30\% des objets connectés de l'IoT ne sont toujours pas sécurisés.
	La sécurité est non-seulement négligée du côté des constructeurs, mais aussi du côté des utilisateurs, la plupart ne prenant pas la
	peine de changer les identifiants par défaut, lorsqu'il y en a. Une pratique pouvant facilement être vérifiable en faisant un tour sur 
	Shodan. Il s'agit d'un genre de moteur de recherche pour objets connectés.
	
	\subsection{Failles systèmes}
	
	\section{Pourquoi c'est dangereux}
	\subsection{BotNet}
	\subsection{Souriez, vous êtes filmés}
	
	\section{Conclusion}
\end{document}